\documentclass[xcolor=dvipsnames,handout,aspectratio=169]{beamer}


\usepackage{natbib}% Recommended
\usepackage{csquotes}
\usepackage{amssymb}
\usepackage{amsfonts}
\usepackage{amsxtra}
\usepackage{amsthm}
\usepackage{scalefnt}
\usepackage{natbib}
\usepackage{listings}

\hypersetup{
    colorlinks,
    citecolor=green,
    linkcolor=red,
    backref=true
}

\let\oldbibitem=\bibitem
\renewcommand{\bibitem}[2][]{\label{#2}\oldbibitem[#1]{#2}}
\let\oldcite=\cite
\renewcommand\cite[1]{\hypersetup{linkcolor=blue} \hyperlink{#1}{\oldcite{#1}}}
\let\oldcitep=\citep
\renewcommand\citep[1]{\hypersetup{linkcolor=blue}\hyperlink{#1}{\oldcitep{#1}}}
\let\oldciteauthor=\citeauthor
\renewcommand\citeauthor[1]{\hypersetup{linkcolor=blue}\hyperlink{#1}{\oldciteauthor{#1}}} 



\setbeamertemplate{navigation symbols}{}
\setbeamertemplate{footline}[frame number]
\DeclareMathOperator{\E}{\mathbb{E}}
\usepackage{tikz}
\usepackage{comment}
\usepackage{pgfplots}
\usepackage{xcolor}
\usetikzlibrary{positioning}
\usetikzlibrary{fit}
\usetikzlibrary{backgrounds}
\usetikzlibrary{calc}
\usetikzlibrary{shapes}
\usetikzlibrary{mindmap}
\usetikzlibrary{patterns}
\usepackage{pifont}
%\usepackage{natbib} 
%\usepackage{bibentry}
\newcommand{\xmark}{\ding{55}}%
\newcommand{\cmark}{\ding{51}}%
\usepackage[skins,theorems]{tcolorbox}
\tcbset{highlight math style={enhanced,
  colframe=red,colback=white,arc=0pt,boxrule=1pt}}

\usetikzlibrary{decorations.text}
\pgfplotsset{compat=1.7}
\mode<presentation>
%\usetheme{}
\usecolortheme[named=Black]{structure}


\usefonttheme{serif}     % Font theme: serif
\usepackage{helvet}     % Font family: Concrete Math
\setbeamersize{text margin left=09mm,text margin right=09mm} 

\colorlet{titleleft}{Sepia!200!black}
\colorlet{titleright}{white}

\setbeamercolor*{frametitle}{fg=black}  %Set frametitle color

\makeatletter
\pgfdeclarehorizontalshading[titleleft,titleright]{beamer@frametitleshade}{\paperheight}{%
  color(0pt)=(titleleft);
  color(\paperwidth)=(titleright)}

\newcommand{\tikzmark}[1]{\tikz[overlay,remember picture] \node (#1) {};}
\setbeamertemplate{enumerate items}{(\arabic{enumi})}
\setbeamertemplate{itemize items}[circle]
\setbeamercovered{transparent=15}
\setbeamercovered{invisible}
\setbeamertemplate{sections/subsections in toc}[sections numbered]




\title[\today]{\textbf{A Course on DSGE Models with Financial Frictions\\ Part 2: Simple RBC \& Dynare Introduction}}
\setbeamercolor{author}{fg=black} 
\author{Stelios Tsiaras}
\institute{European University Institute}
%\date{%16 December 2020}
%\color{blue}{Latvijas Banka}}

\begin{document}


\begin{frame}[noframenumbering]
\titlepage 
\end{frame}

\begin{frame}[c]\frametitle {\textbf{A simple Business Cycle Model}} 
\begin{itemize}
    \setlength\itemsep{2em}
    \item There is a representative household
    \item Chooses optimally consumption and labour subject to its budget constraint
    \item Firms produce output according to a production technology and choose labour and capital inputs to minimize cost
    \item Labour, capital and output markets clear
\end{itemize}
\end{frame}

\begin{frame}[c]\frametitle {\textbf{Summary}} 
\begin{itemize}
    \setlength\itemsep{2em}
\item Households choose hours worked ($H_t$)  and consumption ($C_t$) to maximize their utility
\item Their utility is:
\begin{equation*}\label{Hutil}
U=U(C_t, L_t)
\end{equation*}
where $C_t$ is consumption and all variables are expressed in real terms relative to the price of retail output. We assume that
\begin{equation}\label{Hutilconds}
U_C>0,\,\, U_L>0\,\, U_{CC} \le 0,\,\, U_{LL} \le 0
\end{equation}
\item In a stochastic environment, the \textbf{value function} of the representative household at time $t$ is given by
\begin{equation}\label{HVF}
 V_t=\mathbb{E}_t \left[\sum_{s=0}^\infty \beta^s U(C_{t+s}, L_{t+s})\right]\,; \,\,\beta \in (0,1)
\end{equation}
\end{itemize}
\end{frame}

 \begin{frame}
 \frametitle{\textbf{The Household Optimization Problem}}
  \begin{itemize}
\item Household chooses
$\{C_t\}$, leisure, $\{L_t\}$, labour supply $\{H_t=1-L_t \}$, capital stock $\{K_t\}$ and investment  $\{I_t\}$ to maximize $V_t$ given by \eqref{HVF} given the
budget constraint:
\begin{equation}\label{HBC1}
  B_{t}= R_{t-1} B_{t-1}+r_{t}^K K_{t-1}+W_{t} H_t-C_t-I_t-T_t
\end{equation}
\item $B_t$ is the value of the stock of one-period bonds (price$\times$number of bonds) at the end of period $t$.
\item $r_{t}^K$ is the rental rate for capital, $W_t$ is the wage rate and $R_{t-1}$ is the interest rate set in period $t-1$ paid in period $t$ on bonds held at the end of period $t-1$.
Note $K_t$ is \emph{end-of-period} capital stock.
\item Capital stock accumulates according to
\begin{equation}\nonumber
K_{t}=(1-\delta) K_{t-1}+I_t
\end{equation}
\end{itemize}
\end{frame}

 \begin{frame}
 \frametitle{\textbf{Solution to the Household Optimization Problem}}
  First order conditions are
\begin{eqnarray*}
% \nonumber to remove numbering (before each equation)
%\mbox{Utility}&:& U_t=U(C_t, L_t)\label{RBCLambda}\\
\mbox{Euler Consumption }&:&  R_{t}\mathbb{E}_t\left[ \Lambda_{t,t+1}\right]= 1  \label{RBCEuler}\\
\label{RBCMRS1}
\mbox{Labour Supply}&:&  \frac{U_{H,t}}{U_{C,t}}=-W_t\\
\mbox{Capital Supply}&:& \mathbb{E}_t\left[ \Lambda_{t,t+1} R_{t+1}^K\right]= 1  \label{RBCFOCK1A}
 \end{eqnarray*}
 where the gross return on capital is given by
\begin{equation*}
 R_{t}^K= \left[r_{t}^K +(1-\delta)\right]\label{RK}
\end{equation*}
 and $\Lambda_{t,t+1} \equiv \beta \frac{ U_{C,t+1}}{ U_{C,t}}$ is the \emph{real stochastic discount factor} $[t, t+1]$.
 \end{frame}


 \begin{frame}
 \frametitle{\textbf{Solution to the Household Optimization Problem}}
 \begin{itemize}
\item Then we have the arbitrage condition
 \begin{equation}\label{RBCArbitrage} \nonumber
 1= R_t \mathbb{E}_t\left[ \Lambda_{t,t+1} \right]=\mathbb{E}_t\left[ \Lambda_{t,t+1} R_{t+1}^K\right]
 \end{equation}
 \item In our financial frictions models $R_t \mathbb{E}_t\left[ \Lambda_{t,t+1} \right]\neq \mathbb{E}_t\left[ \Lambda_{t,t+1} R_{t+1}^K\right]$
 \end{itemize}
 \end{frame}

  \begin{frame}
 \frametitle{\textbf{Production Side and Closing the Model}}
 \begin{itemize}
 \item  Output and the firm's behaviour is summarized by:
  \begin{eqnarray}
 % \nonumber to remove numbering (before each equation)
 \mbox{Output}&:& Y_{t}=A_tK_t^{\alpha}H_t^{1-\alpha}\label{RBCPFWhole} \nonumber\\
 \mbox{Labour Demand}&:&  \frac{\alpha Y_{t}}{H_t} = W_t\label{RBCFOCW1} \nonumber\\
 \mbox{Capital Demand}&:& \frac{(1-\alpha) Y_{t}}{K_{t-1}} = r_t^K\label{RBCFOK1} \nonumber
   \end{eqnarray}
\item The model is completed with an output equilibrium and a balanced budget constraint with lump-sum taxes ($T_t$).
 \begin{eqnarray*}
 % \nonumber to remove numbering (before each equation)
  Y_{t} &=&C_{t}+G_t+I_t\\
 \label{GBC1}G_t&=&T_t
 \end{eqnarray*}
where $G_t$ is government spending and $A_t$ follows an AR(1) process: $\ln A_{t}-\ln \bar{A}_{t}=\rho_{A} (\ln A_{t-1}-\ln \bar{A}_{t-1})+\epsilon_{A,t}$
 \end{itemize}
 \end{frame}

\begin{frame}
 \frametitle{\textbf{The Steady State}}
 \begin{itemize}
\item We assume a  \textbf{zero-growth} steady state and a CRRA utility $U = \ln C_t - \omega\frac{H_t^{1+\phi}}{1+\phi}$
where $\omega> 0$ indicates how leisure is valued relative to consumption,
and $\phi> 0$ is the inverse of the labour supply elasticity
\item  $\bar{A}_t=\bar{A}_{t-1}=A$, say and $\bar{G}_t=\bar{G}_{t-1}=G$. $K_t=K_{t-1}=K$  etc
\item  The zero-growth  steady state in recursive form is given by:
\begin{eqnarray*}
R&=&\frac{1}{\beta}\\
\label{KY} \frac{K}{Y}&=&\frac{ (1-\alpha)}{R-1+\delta}  \\
\label{IY}
\frac{I}{Y}&=& \frac{\delta K}{Y}=\frac{ (1-\alpha)\delta}{R-1+\delta} =\frac{ (1-\alpha)\delta}{R-1+\delta}\\
\label{CY}  \frac{C}{Y}&=& 1-\frac{I}{Y}-\frac{G}{Y}=1-\frac{I}{Y}-g_y
\end{eqnarray*}
 \end{itemize}
\end{frame}

 \begin{frame}
  \frametitle{\textbf{Zero Growth Steady State in Recursive Form (cont) }}
 \begin{eqnarray*}
     H &=& \left(\frac{\alpha}{C/Y}\frac{1}{\omega}\right)^{\frac{1}{1+\phi}}\\
        Y&=&   (A H)^{\alpha} K^{1-\alpha}= (A H)^{ \alpha} \left(\frac{K}{Y}\right)^{1-\alpha} (Y)^{1-\alpha} \\  &\Rightarrow& Y= (A H) (K/Y)^\frac{{1-\alpha}}{\alpha}\\
               G&=&g_y Y=T \\
   W&=& \alpha  \frac{Y}{H} \\
   I&=& \frac{I}{Y} Y\, ;\,\, C= \frac{C}{Y} Y\,;\,\,  K= \frac{K}{Y} Y \\
   A&=& 1
\end{eqnarray*}
\end{frame}

\begin{frame}[t]\frametitle{\textbf{Solve the Model with Dynare}}
  \begin{itemize}
     \setlength\itemsep{2em}
  \item Dynare uses a technique called \textbf{perturbation}. For more : \cite{judd1998numerical}
  \item Computes first, second and third order Taylor series approximation of the policy rules \textit{around the steady state} 
  \end{itemize}
  It also: 
    \begin{itemize}
           \setlength\itemsep{2em}
  \item Computes the steady state (numerically) of the model
  \item Computes the solution of deterministic models
  \item Estimates (either by maximum-likelihood(ML) or Bayesian approach) parameters of DSGE models and their distribution
\end{itemize}
\end{frame}


\begin{frame}[t]\frametitle{\textbf{Dynare Starters}}
  \begin{itemize}
         \setlength\itemsep{2em}
  \item It is a big collection of Matlab functions that use Matlab in order to solve the model with perturbation 
  \item We just have to set the external path of Matlab to the Dynare folder
  \item Download it at \url{https://www.dynare.org}
  \item Install it. Go to Matlab on menu File/Set Path to add the path to the Dynare subdirectory (to store all the subroutines), e.g. the path would be set to $c:\backslash dynare\backslash 4.4.y\backslash matlab$
\end{itemize}
\end{frame}

\begin{frame}[t]\frametitle{\textbf{Dynare Model [\texttt{.mod}] file}}
  \begin{itemize}
         \setlength\itemsep{2em}
  \item The .mod file is the file where you write down your DSGE model 
 \item It includes several blocks
   \begin{itemize}
 \item Variable block
 \item Parameter block
\item Parameter values block
 \item Model block
 \item Steady state block
 \item Shocks block
 \item Solution (or estimation) block
 \end{itemize}
Super useful reading: \cite{Adjemianetal2011}
\end{itemize}
\end{frame}

\begin{frame}[t]\frametitle{\textbf{Variables and Parameters Block}}
  \begin{itemize}
     \setlength\itemsep{2em}
    \item \texttt{\textbf{var}} block: Names of the endogenous variables \\
    example: \\ \texttt{var K  C  G  A;}
    \item \texttt{\textbf{varexo}} block: Names of the shocks \\
    example: \\ \texttt{varexo epsA  epsG;}
    \item \texttt{\textbf{parameters}} block: Names of the parameters ; Values of the parameters
    \\
    example: \\ $\texttt{parameters alpha  beta  delta }$; \\
    \texttt{alpha=0.3;} \\
    \texttt{beta=0.99;} \\
    \texttt{delta=0.025;} \\
  \end{itemize}

\end{frame}

\begin{frame}[t]\frametitle{\textbf{Model Block}}
  \begin{itemize}
     \setlength\itemsep{2em}
    \item Starts with \texttt{model;} and ends with \texttt{end;}
    \item Type equations ending with ;
    \begin{itemize}
      \item  \texttt{x(-1)} for predetermined variables. The variable is decided in $t-1$ (predetermined), e.g. the capital stock, write it as \texttt{x(-1)} instead of\texttt{x}
      \item  \texttt{x(+1)} for expectations
    \end{itemize}
    example:  \\ \texttt{K = (1-delta)*K(-1)+I;}
  \end{itemize}
\end{frame}

\begin{frame}[t]\frametitle{\textbf{Shocks Block}}
  \begin{itemize}
\setlength\itemsep{2em}
  \item Starts with \texttt{shocks;} and ends with \texttt{end;} 
  \item In between declare shock standard deviations \\
  example: \\ \texttt{shocks;} \\ \texttt{var epsA; \\ stderr 0.02;}\\ \texttt{end;}
  \item The variances (and covariances) of the shocks are defined within these commands 
  \item Sets the std. error of this exogenous variable = 0.02
  \end{itemize}
\end{frame}

\begin{frame}[t]\frametitle{\textbf{Some info}}
\begin{itemize}
      \setlength\itemsep{2em}
     \item \textbf{Note} that each instruction of the .mod file must be terminated by a semicolon
    \item Also Dynare uses 2 forward slashes (//) to comment out any line (whereas MATLAB uses \%). (Note: for Dynare the two are equivalent!)
     \item There need to be as many equations as your endogenous variables declared (except for optimal policy)
     \item Names are case sensitive
     \item The stability ``Blanchard-Kahn'' conditions are met only if the number of jumpers equals the number of eigenvalues greater than one. (See Topic 2).
  \end{itemize}
\end{frame}

\begin{frame}[t]\frametitle{\textbf{The Steady State Block}}
  \begin{itemize}
     \setlength\itemsep{2em}
    \item It's the most difficult and time consuming part
    \item There are two options
    \begin{itemize}
      \item Let Dynare calculate the steady state (sounds good, does not always work)
      \item Calculate it ourselves and then add this as a Matlab function
    \end{itemize}
  \end{itemize}
\end{frame}

\begin{frame}[t]\frametitle{\textbf{The Steady State Block: Option 1}}
  \begin{itemize}
  \setlength\itemsep{1em}
  \item Dynare solves for the steady state of the model, it just need (good!) initial values 
  \item Starts with \texttt{initval;} and ends with \texttt{end;} 
  \item In between, add initial values for all variables
  \item Initial values can be exact numbers or functions that depend on parameters or steady state variables
  \item Then, \emph{steady} command computes the steady state
  \item If the model is quite complicated and the initial values not close to the truth there will be problems $\rightarrow$ \textbf{Option 2}
 \end{itemize}
\end{frame}

\begin{frame}[t]\frametitle{\textbf{The Steady State Block: Option 2}}
  \begin{itemize}
  \setlength\itemsep{1em}
  \item Find the analytical solution for the steady state 
  \item Import it to a Matlab function doing the computation externally with a Matlab program FILENAME\_steadystate.m
  \item Dynare understands that this function gives the steady state of the model
  \item Needs a specific preamble and ending that is provided in these files
 \end{itemize}
\end{frame}

\begin{frame}[t]\frametitle{\textbf{Solution Block}}
  \begin{itemize}
  \setlength\itemsep{1em}
  \item \texttt{stoch\_simul} starts the solution routine for stochastic models and \texttt{simul} for deterministic simulations \\
  example \\
  \texttt{stoch simul(order=1,IRF=20, periods =10000) ;}
  \item There are many options for the stochastic simulation (see Dynare manual for more)
  \item \emph{periods} - specifies the number of simulation periods
  \item \emph{irf} sets the number of periods for which to compute impulse responses
  \item \emph{$order=1$} sets the order of the Taylor approximation (default is two)
 \end{itemize}
\end{frame}

\begin{frame}[t]\frametitle{\textbf{Solve your Model}}
  \begin{itemize}
  \setlength\itemsep{1em}
  \item Just type in Matlab \texttt{dynare modfilename.mod} 
  \item Dynare output is (among many others):
  \begin{itemize}
      \setlength\itemsep{1em}
    \item Policy rules
    \item Moments 
    \item Impulse response functions
    \item Almost everything is in the folder  \url{oo\_.} You will find it in Matlab's workspace right after the solution takes place
  \end{itemize}
 \end{itemize}
\end{frame}

\begin{frame}[t]\frametitle{\textbf{Exercise - Introducing Investment Adjustment Costs}}
  \begin{itemize}
  \setlength\itemsep{1em}
  \item Same problem as above BUT now we have cost $\Phi(\Xi_t)$ for any investment adjustment 
\begin{eqnarray}  \nonumber 
K_{t}&=&(1-\delta) K_{t-1}+(1-\Phi(\Xi_t))I_t\,\\ \nonumber 
\Xi_t&\equiv&\frac{I_t}{I_{t-1}}\,; \,\,\, S',\, S'' \ge
0\,;\,\,\Phi(1)=\Phi'(1)=0 \label{Kaccum1}
 \end{eqnarray}
\item $I_t$ units of output converts to $(1-\Phi(\Xi_t))I_t$ of new capital sold at a real price $Q_t$ 
\item We now have two constraints and two Lagrange multipliers: $\lambda_t$ \& $\mu_t$. 
\item Express $Q_t$ as $\mu_t/\lambda_t$: the marginal value of capital measured in terms of consumption goods (this is Tobin's $Q$) and express the FOC to Investment in terms of $Q_t$.
 \end{itemize}
\end{frame}

\begin{frame}[t]\frametitle{\textbf{Exercise - Introducing Investment Adjustment Costs}}
  \begin{itemize}
      \setlength\itemsep{1em}
\item For  $\Phi(\Xi_t)=\phi_X (\Xi_t-1)^2$ find the new equilibrium and solve the model in Dynare
\item Do it first by setting  $\phi_X = 2$
\item Construct a [0:4] vector with step=1 grid for $\phi_X$ and show the different impulse responses for {\color{blue}{ouput, consumption, investment and labour hours}} for every different value of $\phi_X$ after a positive TFP shock
\item Important! At each iteration set \texttt{dynare modfile.mod \color{magenta}noclearall}. Dynare does not clear up the workspace after each iteration
 \end{itemize}
\end{frame}


\begin{frame}[containsverbatim]\frametitle{\textbf{Exercise - Introducing Investment Adjustment Costs}}
 What you need
  In Matlab
\begin{lstlisting}[language=Octave]  
% Give to Phix the value of every iteration
phiX   = Phix_value(i); 
% Send the parameter value to Dynare Phix_value       
save Phix_value phiX  
% Run the mod file every time with noclearall option        
dynare RBC1_inv_adj noclearall
\end{lstlisting}

In the parameter section of Dynare 

\begin{lstlisting}[language=Octave]
load Phix_value;
set_param_value('phiX',phiX); 
\end{lstlisting}
\end{frame}





\bibliographystyle{natbib}
\bibliography{/Users/alabama/Documents/GitHub/Financial_Firctions_Course_EUI/Financial-Frictions-Course/course_bibl.bib} 

\end{document}